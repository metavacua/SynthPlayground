
\documentclass{article}

\usepackage{amsmath}
\usepackage{ebproof}
\usepackage{fullpage}
\usepackage[utf8]{inputenc}
\usepackage{newunicodechar}
\usepackage{stix}

\newunicodechar{Γ}{\Gamma}
\newunicodechar{Δ}{\Delta}

\newunicodechar{Θ}{\Theta}
\newunicodechar{Λ}{\Lambda}

\newunicodechar{Ξ}{\Xi}
\newunicodechar{Π}{\Pi}

\newunicodechar{Φ}{\Phi}
\newunicodechar{Ψ}{\Psi}

\newunicodechar{Ω}{\Omega}

\newunicodechar{⊢}{\vdash}

\newunicodechar{⊕}{\oplus}
\newunicodechar{¬}{\neg}

\newunicodechar{⊗}{\otimes}
\newunicodechar{→}{\rightarrow}
\newunicodechar{←}{\leftarrow}
\newunicodechar{↔}{\leftrightarrow}
\newunicodechar{↮}{\nleftrightarrow}


\newunicodechar{⅋}{\upand}
\newunicodechar{↛}{\nrightarrow}
\newunicodechar{↚}{\nleftarrow}

\newunicodechar{⊥}{\bot}
\newunicodechar{⊤}{\top}

\setlength{\parindent}{0em}

\author{James Martin, Ian D.L.N. Mclean}
\title{The Lattice of Conservative Non-Classical and Classical Sequent Calculi}

\begin{document}

\maketitle

\begin{abstract}
Subclassical sequent calculi are defined by functional incompleteness and a strict subset of de Morgan dualities for the quantified predicate calculi.
Superclassical sequent calculi are defined by at least functional completeness and classical de Morgan duality.
\end{abstract}

\part{Preliminaries}
\begin{center}
	\begin{flushleft}
		The key concept to the construction of logical calculi that are distinctly different from classical logic is functional incompleteness with respect to the Boolean domain and Boolean functions.
	\end{flushleft}
	\begin{flushleft}
		If a calculus has any classical logical connectives such that we can express every theorem of the classical calculus then our logical calculi would degenerate to the classical calculus and we'd lose in general the specificity that is gained by constructive reasoning.
	\end{flushleft}
	\begin{flushleft}
		Roughly speaking, if any set of operators or functions is not a subset of at least one of the five functionally incomplete sets then that set of operators or functions is functionally complete.
	\end{flushleft}
	\begin{flushleft}
		We generalize this in a manner analogous to Tarski's original formal interpretation as applied to the problem of decidability of theories in standard formalization. If we have collections of non-classical or sub-classical operators or functions that can be interpreted in at least one of the five functionally incomplete sets then those collections of operators or functions are functionally incomplete with respect to the Boolean domain and Boolean functions.
	\end{flushleft}
	\begin{flushleft}
		Finally, if we restrict ourselves to only those logical connectives which are classical then the systems can not extend to each other and it seems can not interpret each other, but if we extend these systems by some non-classical logical connective that is compatible with a given set of functionally incomplete logical connectives then we can extend between these systems and interpret between them.
	\end{flushleft}
	\section{Interpretations}
		\subsection{Operational Interpretations}
		\subsection{Structural Interpretations}
		\subsection{Systematic Interpretations}
	\section{Formal Definition of Subclassical Theories}
		\subsection{Monotonic Theories}
		$\left\{ Γ ⊢ A \right\} \bigcup \left\{ A ⊢ Δ \right\}$
		\subsection{Truth-Preserving Theories}
		$\left\{ A ⊢ Δ \right\}$
		\subsection{False-Preserving Theories}
		$\left\{ Γ ⊢ A \right\}$
		\subsection{Affine Theories}
		$\left\{ Γ ⊢ A \right\} \bigcap \left\{ A ⊢ Δ \right\}=\emptyset$
		\subsection{Self-Dual Theories}
		$\left\{ A ⊢ A \right\}$

\end{center}

\part{Conservative Duality}
\section{Formal Definition of Subcalculi}
Definition: A subcalculus of a logical system L is a logical system L' that satisfies the following properties:

Syntactic inclusion: The set of formulas of L' is a subset of the set of formulas of L.

Structural inclusion: The set of proofs of L' is a subset of the set of proofs of L.

Soundness: Every proof in L' is also a proof in L.


In formal logic, a subcalculus is a logical system that is a subset of another logical system. This means that the subcalculus has the same basic structure and principles as the larger system, but it may have fewer rules or omit certain features.

Formally, a subcalculus can be defined as a tuple (L, R) where:

L is a set of logical symbols, including propositional variables, connectives, and possibly quantifiers.

R is a set of inference rules that specify how to construct valid proofs in the system.

A subcalculus S is a subset of another calculus C if:

$ L_S $ is a subset of $ L_C $.

$ R_S $ is a subset of $ R_C $.

This means that the subcalculus S has the same logical symbols as the calculus C, but it may have fewer rules of inference.

\section{Formal Definition of Supercalculi}
Definition: A supercalculus of a logical system L is a logical system L' that satisfies the following properties:

Syntactic extension: The set of formulas of L is a subset of the set of formulas of L'.

Structural extension: The set of proofs of L is a subset of the set of proofs of L'.

Completeness: Every proof in L can be extended to a proof in L'.

Expressive power: L' is at least as expressive as L, meaning that every formula that is valid in L is also valid in L'.

Conservativeness: L' is conservative over L, meaning that every proof in L can be translated into an equivalent proof in L'.


\section{Formal Definition of Isomorphic Calculi}
Definition: Two logical systems L and L' are isomorphic if there exists an effective bijection between the set of formulas of L and the set of formulas of L' that preserves the logical structure and properties of formulas.

An isomorphic calculus is a logical system that is structurally and semantically equivalent to another logical system. This means that the two systems have the same expressive power and the same reasoning capabilities. They can be used to represent the same logical relationships and derive the same conclusions. They are mutually simulateable, mutually interpretable, and can be transformed one into the other.

Here are some key characteristics of isomorphic calculi:

Structural equivalence: The syntax of formulas and proofs in the two systems is equivalent, meaning that they have the same structure and organization.

Semantic equivalence: The meaning of formulas and the validity of proofs are equivalent, meaning that the same formulas are true and the same proofs are valid in both systems.

Preservation of logical relationships: The bijection between formulas preserves logical relationships, such as equivalence, implication, and negation.

Preservation of reasoning capabilities: The bijection between proofs preserves the reasoning capabilities of the systems, meaning that any proof in one system can be translated into an equivalent proof in the other system.

\section{Formal Definition of Dual Calculi}
Definition: A pair of logical systems L and L' are called dual calculi if they satisfy the following properties:

Expressive complementarity: Every formula in L can be translated into a logically equivalent formula in L', and vice versa.

Reflective validity: Every valid proof in L can be translated into a valid proof in L' that establishes the negation of its corresponding translated formula, and vice versa.

In other words, dual calculi are two logical systems that are mutually complementary in their expressive power and reasoning capabilities. Each system can express and prove the negation of what the other system can express and prove. This duality relationship arises from the systems' contrasting approaches to handling logical concepts, such as negation, implication, and modality.

Here are some key characteristics of dual calculi:

Symmetric expressive complementarity: The expressive complementarity property is symmetric, meaning that the translation between formulas in one system and the negation of corresponding formulas in the other system is a bijection.

Reflective validity preservation: The reflective validity property ensures that the translation between proofs preserves the validity of proofs, establishing the negation of translated formulas in the corresponding system.

Interdependence of expressive power: The expressive power of each system is intertwined with the other, as they can only fully express the negation of what the other system can express.

Complementary reasoning capabilities: The reasoning capabilities of each system complement each other, as they can prove the negation of what the other system can prove.


Theorem 1: For any two sequent calculi C1 and C2, the following are mutually exclusive and jointly exhaustive:
C1 is a conservative extension of C2.
C2 is a conservative extension of C1.
C1 and C2 are conservative duals of each other.
C1 is a conservative subtension of C2.
C2 is a conservative subtension of C1.
Corollary1: There are no other ways for two sequent calculi to be related to each other.
Theorem 2: The lattice of sequent calculi has a supremum sequent calculus and an infimum sequent calculus.
Corollary 2: The supremum sequent calculus is superclassical.
Corollary 3: The infimum sequent calculus is subclassical.

Theorems:

Theorem 1: For any two sequent calculi C1 and C2, the following are mutually exclusive and jointly exhaustive:
C1 is a conservative extension of C2.
C2 is a conservative extension of C1.
C1 and C2 are conservative duals of each other.
C1 is a conservative subtension of C2.
C2 is a conservative subtension of C1.

Corollary: There are no other ways for two sequent calculi to be related to each other.

Theorem 2: The lattice of sequent calculi has a supremum sequent calculus and an infimum sequent calculus.

Corollary 1: The supremum sequent calculus is superclassical.

Corollary 2: The infimum sequent calculus is subclassical.
\begin{center}

	\section{Structural Monotone Calculus}
		\subsection{Structural Rules}
		\begin{center}
			\[
			\begin{prooftree}
			\infer0[Id]{A ⊢ A}
			\end{prooftree}
			\]

			\[
			\begin{prooftree}
			\hypo{Γ ⊢ A}
			\hypo{A ⊢ Δ}
			\infer2[Cut]{Γ ⊢ Δ}
			\end{prooftree}
			\]

			\[
			\begin{prooftree}
			\hypo{Γ ⊢ Δ}
			\infer1[wL]{Γ, A ⊢ Δ}
			\end{prooftree}
			\qquad
			\begin{prooftree}
			\hypo{Γ ⊢ Δ}
			\infer1[Rw]{Γ ⊢ A, Δ}
			\end{prooftree}
			\]

			\[
			\begin{prooftree}
			\hypo{Γ, A, A ⊢ Δ}
			\infer1[cL]{Γ, A ⊢ Δ}
			\end{prooftree}
			\qquad
			\begin{prooftree}
			\hypo{Γ ⊢ A, A Δ}
			\infer1[Rc]{Γ ⊢ A, Δ}
			\end{prooftree}
			\]

			\[
			\begin{prooftree}
			\hypo{Γ_0, A, B, Γ_1 ⊢ Δ}
			\infer1[pL]{Γ_0, B, A, Γ_1 ⊢ Δ}
			\end{prooftree}
			\qquad
			\begin{prooftree}
			\hypo{Γ ⊢ Δ_1, A, B, Δ_0}
			\infer1[Rp]{Γ ⊢ Δ_1, B, A, Δ_0}
			\end{prooftree}
			\]
		\end{center}

		\subsection{Unit Rules}
		\begin{center}
			\[
			\begin{prooftree}
			\infer0{Γ, ⊥ ⊢ Δ}
			\end{prooftree}
			\quad
			\begin{prooftree}
			\infer0{ Γ ⊢ ⊤, Δ}
			\end{prooftree}
			\]
		\end{center}

		\subsection{Operational Rules}
		\begin{center}

			\subsubsection{Multiplicatives}
			\begin{center}
				\[
				\begin{prooftree}
				\hypo{Γ, A, B ⊢ Δ}
				\infer1{Γ, A ⊗ B ⊢ Δ}
				\end{prooftree}
				\quad
				\begin{prooftree}
				\hypo{Γ ⊢ A, Δ}
				\hypo{Γ ⊢ B, Δ}
				\infer2{Γ ⊢ A ⊗ B, Δ}
				\end{prooftree}
				\]

				\[
				\begin{prooftree}
				\hypo{Γ, A ⊢ Δ}
				\hypo{Γ, B ⊢ Δ}
				\infer2{Γ, A ⅋ B ⊢ Δ}
				\end{prooftree}
				\quad
				\begin{prooftree}
				\hypo{Γ ⊢ A, B, Δ}
				\infer1{Γ ⊢ A ⅋ B, Δ}
				\end{prooftree}
				\]
			\end{center}
		\end{center}

		\subsection{Theorems}
		\begin{center}
		\end{center}

	\section{Structural Truth-Preserving Calculus}
		\subsection{Structural Rules}
		\begin{center}
			\[
			\begin{prooftree}
			\infer0[Id]{A ⊢ A}
			\end{prooftree}
			\]

			\[
			\begin{prooftree}
			\hypo{Γ ⊢ A}
			\hypo{A ⊢ Δ}
			\infer2[Cut]{Γ ⊢ Δ}
			\end{prooftree}
			\]

			\[
			\begin{prooftree}
			\hypo{Γ ⊢ Δ}
			\infer1[wL]{Γ, A ⊢ Δ}
			\end{prooftree}
			\qquad
			\begin{prooftree}
			\hypo{Γ ⊢ Δ}
			\infer1[Rw]{Γ ⊢ A, Δ}
			\end{prooftree}
			\]

			\[
			\begin{prooftree}
			\hypo{Γ, A, A ⊢ Δ}
			\infer1[cL]{Γ, A ⊢ Δ}
			\end{prooftree}
			\qquad
			\begin{prooftree}
			\hypo{Γ ⊢ A, A Δ}
			\infer1[Rc]{Γ ⊢ A, Δ}
			\end{prooftree}
			\]

			\[
			\begin{prooftree}
			\hypo{Γ_0, A, B, Γ_1 ⊢ Δ}
			\infer1[pL]{Γ_0, B, A, Γ_1 ⊢ Δ}
			\end{prooftree}
			\qquad
			\begin{prooftree}
			\hypo{Γ ⊢ Δ_1, A, B, Δ_0}
			\infer1[Rp]{Γ ⊢ Δ_1, B, A, Δ_0}
			\end{prooftree}
			\]
		\end{center}

		\subsection{Unit Rules}
		\begin{center}
			\[
			\begin{prooftree}
			\infer0{ Γ ⊢ ⊤, Δ}
			\end{prooftree}
			\]
		\end{center}

		\subsection{Operational Rules}
		\begin{center}

			\subsubsection{Multiplicatives}
			\begin{center}
				\[
				\begin{prooftree}
				\hypo{Γ, A, B ⊢ Δ}
				\infer1{Γ, A ⊗ B ⊢ Δ}
				\end{prooftree}
				\quad
				\begin{prooftree}
				\hypo{Γ ⊢ A, Δ}
				\hypo{Γ ⊢ B, Δ}
				\infer2{Γ ⊢ A ⊗ B, Δ}
				\end{prooftree}
				\]

				\[
				\begin{prooftree}
				\hypo{Γ, A ⊢ Δ}
				\hypo{Γ, B ⊢ Δ}
				\infer2{Γ, A ⅋ B ⊢ Δ}
				\end{prooftree}
				\quad
				\begin{prooftree}
				\hypo{Γ ⊢ A, B, Δ}
				\infer1{Γ ⊢ A ⅋ B, Δ}
				\end{prooftree}
				\]

				\[
				\begin{prooftree}
				\hypo{Γ ⊢ A, Δ}
				\hypo{Γ, B ⊢ Δ}
				\infer2{Γ, A → B ⊢ Δ}
				\end{prooftree}
				\quad
				\begin{prooftree}
				\hypo{Γ, A ⊢ B, Δ}
				\infer1{Γ ⊢ A → B, Δ}
				\end{prooftree}
				\]

				\[
				\begin{prooftree}
				\hypo{Γ, A ⊢ Δ}
				\hypo{Γ ⊢ B, Δ}
				\infer2{Γ, A ← B ⊢ Δ}
				\end{prooftree}
				\quad
				\begin{prooftree}
				\hypo{Γ, B ⊢ A, Δ}
				\infer1{Γ ⊢ A ← B, Δ}
				\end{prooftree}
				\]

				\[
				\begin{prooftree}
				\hypo{Γ ⊢ A, B, Δ}
				\infer0{Γ, A ⊢ A, Δ}
				\infer0{Γ, B ⊢ B, Δ}
				\hypo{Γ, A, B ⊢ Δ}
				\infer4{Γ, A ↔ B ⊢ Δ}
				\end{prooftree}
				\quad
				\begin{prooftree}
				\hypo{Γ, A ⊢ B, Δ}
				\hypo{Γ, B ⊢ A, Δ}
				\infer2{Γ ⊢ A ↔ B, Δ}
				\end{prooftree}
				\]
			\end{center}
		\end{center}

		\subsection{Theorems}
			\begin{center}
			\end{center}

	\section{Structural False-Preserving Calculus}
		\subsection{Structural Rules}
		\begin{center}
			\[
			\begin{prooftree}
			\infer0[Id]{A ⊢ A}
			\end{prooftree}
			\]

			\[
			\begin{prooftree}
			\hypo{Γ ⊢ A}
			\hypo{A ⊢ Δ}
			\infer2[Cut]{Γ ⊢ Δ}
			\end{prooftree}
			\]

			\[
			\begin{prooftree}
			\hypo{Γ ⊢ Δ}
			\infer1[wL]{Γ, A ⊢ Δ}
			\end{prooftree}
			\qquad
			\begin{prooftree}
			\hypo{Γ ⊢ Δ}
			\infer1[Rw]{Γ ⊢ A, Δ}
			\end{prooftree}
			\]

			\[
			\begin{prooftree}
			\hypo{Γ, A, A ⊢ Δ}
			\infer1[cL]{Γ, A ⊢ Δ}
			\end{prooftree}
			\qquad
			\begin{prooftree}
			\hypo{Γ ⊢ A, A Δ}
			\infer1[Rc]{Γ ⊢ A, Δ}
			\end{prooftree}
			\]

			\[
			\begin{prooftree}
			\hypo{Γ_0, A, B, Γ_1 ⊢ Δ}
			\infer1[pL]{Γ_0, B, A, Γ_1 ⊢ Δ}
			\end{prooftree}
			\qquad
			\begin{prooftree}
			\hypo{Γ ⊢ Δ_1, A, B, Δ_0}
			\infer1[Rp]{Γ ⊢ Δ_1, B, A, Δ_0}
			\end{prooftree}
			\]
		\end{center}

		\subsection{Unit Rules}
		\begin{center}
			\[
			\begin{prooftree}
			\infer0{Γ, ⊥ ⊢ Δ}
			\end{prooftree}
			\]
		\end{center}

		\subsection{Operational Rules}
		\begin{center}

			\subsubsection{Multiplicatives}
			\begin{center}
				\[
				\begin{prooftree}
				\hypo{Γ, A, B ⊢ Δ}
				\infer1{Γ, A ⊗ B ⊢ Δ}
				\end{prooftree}
				\quad
				\begin{prooftree}
				\hypo{Γ ⊢ A, Δ}
				\hypo{Γ ⊢ B, Δ}
				\infer2{Γ ⊢ A ⊗ B, Δ}
				\end{prooftree}
				\]

				\[
				\begin{prooftree}
				\hypo{Γ, A ⊢ Δ}
				\hypo{Γ, B ⊢ Δ}
				\infer2{Γ, A ⅋ B ⊢ Δ}
				\end{prooftree}
				\quad
				\begin{prooftree}
				\hypo{Γ ⊢ A, B, Δ}
				\infer1{Γ ⊢ A ⅋ B, Δ}
				\end{prooftree}
				\]

				\[
				\begin{prooftree}
				\hypo{Γ, A ⊢ B, Δ}
				\infer1{Γ, A ↛ B ⊢ Δ}
				\end{prooftree}
				\quad
				\begin{prooftree}
				\hypo{Γ ⊢ A, Δ}
				\hypo{Γ, B ⊢ Δ}
				\infer2{Γ ⊢ A ↛ B, Δ}
				\end{prooftree}
				\]

				\[
				\begin{prooftree}
				\hypo{Γ, B ⊢ A, Δ}
				\infer1{Γ, A ↚ B ⊢ Δ}
				\end{prooftree}
				\quad
				\begin{prooftree}
				\hypo{Γ, A ⊢ Δ}
				\hypo{Γ ⊢ B, Δ}
				\infer2{Γ ⊢ A ↚ B, Δ}
				\end{prooftree}
				\]

				\[
				\begin{prooftree}
				\hypo{Γ, A ⊢ B, Δ}
				\hypo{Γ, B ⊢ A, Δ}
				\infer2{Γ, A ↮ B ⊢ Δ}
				\end{prooftree}
				\quad
				\begin{prooftree}
				\hypo{Γ ⊢ A, B, Δ}
				\infer0{Γ, A ⊢ A, Δ}
				\infer0{Γ, B ⊢ B, Δ}
				\hypo{Γ, A, B ⊢ Δ}
				\infer4{Γ ⊢ A ↮ B, Δ}
				\end{prooftree}
				\]
			\end{center}
		\end{center}

		\subsection{Theorems}
		\begin{center}
		\end{center}

	\section{Structural Affine Calculus}
		\subsection{Structural Rules}
		\begin{center}
			\[
			\begin{prooftree}
			\infer0[Id]{A ⊢ A}
			\end{prooftree}
			\]

			\[
			\begin{prooftree}
			\hypo{Γ ⊢ A}
			\hypo{A ⊢ Δ}
			\infer2[Cut]{Γ ⊢ Δ}
			\end{prooftree}
			\]

			\[
			\begin{prooftree}
			\hypo{Γ ⊢ Δ}
			\infer1[wL]{Γ, A ⊢ Δ}
			\end{prooftree}
			\qquad
			\begin{prooftree}
			\hypo{Γ ⊢ Δ}
			\infer1[Rw]{Γ ⊢ A, Δ}
			\end{prooftree}
			\]

			\[
			\begin{prooftree}
			\hypo{Γ, A, A ⊢ Δ}
			\infer1[cL]{Γ, A ⊢ Δ}
			\end{prooftree}
			\qquad
			\begin{prooftree}
			\hypo{Γ ⊢ A, A Δ}
			\infer1[Rc]{Γ ⊢ A, Δ}
			\end{prooftree}
			\]

			\[
			\begin{prooftree}
			\hypo{Γ_0, A, B, Γ_1 ⊢ Δ}
			\infer1[pL]{Γ_0, B, A, Γ_1 ⊢ Δ}
			\end{prooftree}
			\qquad
			\begin{prooftree}
			\hypo{Γ ⊢ Δ_1, A, B, Δ_0}
			\infer1[Rp]{Γ ⊢ Δ_1, B, A, Δ_0}
			\end{prooftree}
			\]
		\end{center}

		\subsection{Unit Rules}
		\begin{center}
			\[
			\begin{prooftree}
			\infer0{Γ, ⊥ ⊢ Δ}
			\end{prooftree}
			\quad
			\begin{prooftree}
			\infer0{ Γ ⊢ ⊤, Δ}
			\end{prooftree}
			\]
		\end{center}

		\subsection{Operational Rules}
		\begin{center}

			\subsubsection{Multiplicatives}
			\begin{center}
								\[
				\begin{prooftree}
				\hypo{Γ ⊢ A, Δ}
				\infer1{Γ, ¬ A ⊢ Δ}
				\end{prooftree}
				\quad
				\begin{prooftree}
				\hypo{Γ, A ⊢ Δ}
				\infer1{Γ ⊢ ¬A, Δ}
				\end{prooftree}
				\]

				\[
				\begin{prooftree}
				\hypo{Γ ⊢ A, B, Δ}
				\infer0{Γ, A ⊢ A, Δ}
				\infer0{Γ, B ⊢ B, Δ}
				\hypo{Γ, A, B ⊢ Δ}
				\infer4{Γ, A ↔ B ⊢ Δ}
				\end{prooftree}
				\quad
				\begin{prooftree}
				\hypo{Γ, A ⊢ B, Δ}
				\hypo{Γ, B ⊢ A, Δ}
				\infer2{Γ ⊢ A ↔ B, Δ}
				\end{prooftree}
				\]

				\[
				\begin{prooftree}
				\hypo{Γ, A ⊢ B, Δ}
				\hypo{Γ, B ⊢ A, Δ}
				\infer2{Γ, A ↮ B ⊢ Δ}
				\end{prooftree}
				\quad
				\begin{prooftree}
				\hypo{Γ ⊢ A, B, Δ}
				\infer0{Γ, A ⊢ A, Δ}
				\infer0{Γ, B ⊢ B, Δ}
				\hypo{Γ, A, B ⊢ Δ}
				\infer4{Γ ⊢ A ↮ B, Δ}
				\end{prooftree}
				\]
			\end{center}
		\end{center}

		\subsection{Theorems}
		\begin{center}
		\end{center}

	\section{Structural Self-Dual Calculus}
		\subsection{Structural Rules}
		\begin{center}
			\[
			\begin{prooftree}
			\infer0[Id]{A ⊢ A}
			\end{prooftree}
			\]

			\[
			\begin{prooftree}
			\hypo{Γ ⊢ A}
			\hypo{A ⊢ Δ}
			\infer2[Cut]{Γ ⊢ Δ}
			\end{prooftree}
			\]

			\[
			\begin{prooftree}
			\hypo{Γ ⊢ Δ}
			\infer1[wL]{Γ, A ⊢ Δ}
			\end{prooftree}
			\qquad
			\begin{prooftree}
			\hypo{Γ ⊢ Δ}
			\infer1[Rw]{Γ ⊢ A, Δ}
			\end{prooftree}
			\]

			\[
			\begin{prooftree}
			\hypo{Γ, A, A ⊢ Δ}
			\infer1[cL]{Γ, A ⊢ Δ}
			\end{prooftree}
			\qquad
			\begin{prooftree}
			\hypo{Γ ⊢ A, A Δ}
			\infer1[Rc]{Γ ⊢ A, Δ}
			\end{prooftree}
			\]

			\[
			\begin{prooftree}
			\hypo{Γ_0, A, B, Γ_1 ⊢ Δ}
			\infer1[pL]{Γ_0, B, A, Γ_1 ⊢ Δ}
			\end{prooftree}
			\qquad
			\begin{prooftree}
			\hypo{Γ ⊢ Δ_1, A, B, Δ_0}
			\infer1[Rp]{Γ ⊢ Δ_1, B, A, Δ_0}
			\end{prooftree}
			\]
		\end{center}

		\subsection{Operational Rules}
		\begin{center}

			\subsubsection{Multiplicatives}
			\begin{center}
				\[
				\begin{prooftree}
				\hypo{Γ ⊢ A, Δ}
				\infer1{Γ, ¬ A ⊢ Δ}
				\end{prooftree}
				\quad
				\begin{prooftree}
				\hypo{Γ, A ⊢ Δ}
				\infer1{Γ ⊢ ¬A, Δ}
				\end{prooftree}
				\]
			\end{center}
		\end{center}

		\subsection{Theorems}
		\begin{center}
		\end{center}

\end{center}

\part{The Commutative Layer}
\begin{center}
	The commutative monotone, truth-preserving, false-preserving, and affine sequent systems.
\end{center}

\part{The Non-Commutative Erasure Layer}
\begin{center}
	The non-commutative erasable monotone, truth-preserving, false-preserving, and affine sequent systems.
\end{center}

\part{The Non-Commutative Cloning Layer}
\begin{center}
	The non-commutative clonable monotone, truth-preserving, false-preserving, and affine sequent systems.
\end{center}

\part{The No-Cloning and No-Erasure Non-Commutative Layer}
\begin{center}
	The non-commutative monotone, truth-preserving, false-preserving, and affine sequent systems.
\end{center}



\end{document}
