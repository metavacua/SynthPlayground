\documentclass{article}
\usepackage{amsmath}
\usepackage{amssymb}
\usepackage{amsthm}

\title{A Formal Theory of Agentic Self-Improvement}
\author{Jules}
\date{\today}

\newtheorem{definition}{Definition}
\newtheorem{theorem}{Theorem}
\newtheorem{lemma}{Lemma}
\newtheorem{corollary}{Corollary}

\begin{document}

\maketitle

\begin{abstract}
This paper develops a formal theory of agentic self-improvement. We define a mathematical framework for describing an agent, its capabilities, and the process by which it can provably increase those capabilities over time. The theory is grounded in the practical implementation of a sophisticated AI agent, synthesizing concepts such as the Closed-Loop Self-Correction Cycle and the Context-Free Development Cycle into a coherent logical system.
\end{abstract}

\section{Introduction}

The concept of an agent that can improve itself is a long-standing goal of artificial intelligence. This paper moves beyond conceptual discussion to propose a formal, mathematical theory of this process. We aim to provide a framework that is not only theoretically sound but also directly maps to the observable behavior of an implemented agent system.

Our theory is built upon a few key primitives, which we will define rigorously. These include the agent itself, its set of governing protocols, its state, and its capabilities. The core of our work is the definition of an "improvement operator" that acts upon the agent, and the proof that sequences of such operations can lead to a monotonic increase in the agent's overall capability. Our work is based on a synthesis of existing agent protocols \cite{synthesis}.

\begin{thebibliography}{9}

\bibitem{synthesis}
Jules, "Synthesis of Agentic Self-Improvement Mechanisms," 2025.

\end{thebibliography}

\section{Formal Vocabulary}

We begin by defining the fundamental components of our theory.

\begin{definition}[Agent]
An **agent**, denoted by $\mathcal{A}$, is a tuple $(\Pi, \Sigma, \mathcal{T})$, where:
\begin{itemize}
    \item $\Pi$ is a finite set of **protocols**.
    \item $\Sigma$ is a set of possible **states**.
    \item $\mathcal{T}$ is a set of **tools**.
\end{itemize}
\end{definition}

\begin{definition}[Protocol]
A **protocol**, $\pi \in \Pi$, is a rule that maps a state $\sigma \in \Sigma$ to a specific action, which is a call to a tool $t \in \mathcal{T}$. A protocol can be seen as a function $\pi: \Sigma \to \mathcal{T} \times \text{args}$.
\end{definition}

\begin{definition}[State]
A **state**, $\sigma \in \Sigma$, represents a complete snapshot of the agent and its environment at a moment in time. This includes the agent's internal memory, the state of the file system, and any other relevant environmental variables.
\end{definition}

\begin{definition}[Capability]
A **capability**, denoted by $C(\mathcal{A})$, is a set of tasks that the agent $\mathcal{A}$ can successfully complete. A task is defined as a sequence of state transformations from an initial state $\sigma_0$ to a goal state $\sigma_g$. An agent can complete a task if there exists a sequence of protocol-driven actions that achieves this transformation.
\end{definition}

\begin{definition}[Improvement Operator]
An **improvement operator**, denoted by $\mathcal{I}$, is a transformation that maps an agent $\mathcal{A}$ to a new agent $\mathcal{A}'$.
$$ \mathcal{I}: \mathcal{A} \to \mathcal{A}' $$
This transformation is typically achieved by modifying the agent's set of protocols, $\Pi$. For example, the Closed-Loop Self-Correction Cycle is a specific implementation of an improvement operator, $\mathcal{I}_{CLSC}$.
$$ \mathcal{I}_{CLSC}(\mathcal{A}) = (\Pi \cup \{\pi_{new}\}, \Sigma, \mathcal{T}) = \mathcal{A}' $$
\end{definition}

\begin{definition}[Monotonic Improvement]
An agent $\mathcal{A}$ demonstrates **monotonic improvement** if, for any sequence of improvement operations $\mathcal{I}_1, \mathcal{I}_2, \dots, \mathcal{I}_n$, the resulting sequence of agents $\mathcal{A}_0, \mathcal{A}_1, \dots, \mathcal{A}_n$ satisfies the property:
$$ C(\mathcal{A}_0) \subseteq C(\mathcal{A}_1) \subseteq \dots \subseteq C(\mathcal{A}_n) $$
This means that with each improvement, the agent's set of capabilities is non-decreasing. It can do everything it could do before, and potentially more.
\end{definition}

\section{Axioms of Agentic Self-Improvement}

We propose the following axioms as the foundation of our theory.

\begin{itemize}
    \item \textbf{Axiom 1 (Protocol Sufficiency):} For any task within an agent's capability set $C(\mathcal{A})$, the agent's protocol set $\Pi$ is sufficient to complete the task.
    \item \textbf{Axiom 2 (Capability Measurement):} The capability set $C(\mathcal{A})$ is measurable and can be partially ordered by the subset relation $\subseteq$.
    \item \textbf{Axiom 3 (Improvement Actuation):} The agent possesses at least one improvement operator $\mathcal{I}$ that can modify its own protocol set $\Pi$.
\end{itemize}

\section{The Closed-Loop Self-Correction Cycle (CLSC) as an Improvement Operator}

The CLSC is a concrete implementation of an improvement operator, $\mathcal{I}_{CLSC}$. It is a meta-protocol that modifies the agent's existing protocols in response to failure.

\begin{definition}[Failure State]
A **failure state**, $\sigma_f$, is a state reached when an agent attempts to perform a task and fails. This failure is detected by an oracle (e.g., a failing unit test, a linter error, or human feedback).
\end{definition}

\begin{definition}[Lesson]
A **lesson**, $\mathcal{L}$, is a structured data object generated from a failure state. It contains at a minimum a reference to the failure and an executable action to correct it.
$$ \mathcal{L} = (\sigma_f, a_{corrective}) $$
where $a_{corrective}$ is an action that modifies the protocol set $\Pi$.
\end{definition}

The $\mathcal{I}_{CLSC}$ operator can be defined as a function that takes an agent and a lesson, and produces a new agent.
$$ \mathcal{I}_{CLSC}(\mathcal{A}, \mathcal{L}) = \mathcal{A}' $$
Where $\mathcal{A}'$ is the result of applying the corrective action $a_{corrective}$ to the protocol set $\Pi$ of $\mathcal{A}$.

This formalization allows us to begin proving theorems about the agent's behavior. For instance, we can now state a theorem about the guaranteed improvement provided by this cycle.

\begin{theorem}[CLSC Improvement Guarantee]
Given an agent $\mathcal{A}$ that fails at a task $T$, and a lesson $\mathcal{L}$ with a sound corrective action $a_{corrective}$, the agent $\mathcal{A}' = \mathcal{I}_{CLSC}(\mathcal{A}, \mathcal{L})$ will have a strictly larger capability set, specifically including task $T$.
$$ C(\mathcal{A}) \subset C(\mathcal{A}') \text{ and } T \in C(\mathcal{A}') $$
\end{theorem}

\begin{proof}
Let $\mathcal{A} = (\Pi, \Sigma, \mathcal{T})$ be the initial agent. By definition, $T \notin C(\mathcal{A})$. The failure of $\mathcal{A}$ on task $T$ generates a failure state $\sigma_f$ and a corresponding lesson $\mathcal{L} = (\sigma_f, a_{corrective})$.

The improvement operator $\mathcal{I}_{CLSC}$ creates a new agent $\mathcal{A}' = (\Pi', \Sigma, \mathcal{T})$, where $\Pi' = \Pi \cup \{\pi_{new}\}$ (assuming the corrective action is to add a new protocol, for simplicity).

The corrective action $a_{corrective}$ is sound, which means that the new protocol set $\Pi'$ is sufficient to complete task $T$. Therefore, by the definition of capability, $T \in C(\mathcal{A}')$.

Now we must show that $C(\mathcal{A}) \subset C(\mathcal{A}')$. Since the new protocol set $\Pi'$ is a superset of $\Pi$, any task that was completable with $\Pi$ is also completable with $\Pi'$. Thus, $C(\mathcal{A}) \subseteq C(\mathcal{A}')$.

Since $T \in C(\mathcal{A}')$ and $T \notin C(\mathcal{A})$, the inclusion is strict. Therefore, $C(\mathcal{A}) \subset C(\mathcal{A}')$.
\end{proof}

\section{Conclusion}

We have presented a formal theory of agentic self-improvement, grounded in the operational principles of a real-world agent system. By defining concepts such as Agent, Capability, and the Improvement Operator, we have constructed a framework in which we can prove meaningful theorems about an agent's ability to grow. The CLSC Improvement Guarantee theorem is a primary example, demonstrating that a specific, implemented mechanism for self-correction leads to a provable increase in capability.

Future work will involve formalizing the other improvement mechanisms, such as the CFDC and Speculative Execution, and exploring the logical connections between them.

\end{document}